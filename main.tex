\documentclass[a4paper]{article}
\usepackage[utf8]{inputenc}
\usepackage[T1]{fontenc}
\usepackage[swedish]{babel}
\pagenumbering{arabic}
\usepackage{graphicx}
\usepackage{fancyhdr}
\usepackage{fullpage}
\usepackage{array}
\usepackage{ifpdf}
\usepackage{xcolor}
\unitlength=1cm
\ifpdf\DeclareGraphicsExtensions{.pdf,.jpg}\else\DeclareGraphicsExtensions{.eps}\fi

%%%% Fina headers är bra skit. %%%%
\pagestyle{fancy}
\headheight 35pt
\headsep 40pt
\addtolength{\textheight}{-65pt}
%%%% Välj logga (styretlogo, sektionslogo eller din funktionärslogo) och ändra
\fancyhead[L]{\includegraphics[height=5\baselineskip]{sektionslogo}}
\fancyhead[R]{Kärnstyret\\ \today\\ \emph{Fysikteknologsektionen, Chalmers}}

%%%% Här börjar det riktiga dokumentet %%%%

%För att denna profil ska bli så effektiv som möjligt föreslår vi att ni till en början skriver ner alla egenskaper och erfarenheter som ni anser är viktiga eller kan vara till stor fördel för att driva er förening framåt det kommande året. 

%I er profil vill vi gärna se att har med egenskaper och tidigare erfarenheter som ni anser viktiga för respektive post, samt vad som är viktigt för gruppen i sin helhet. Försök att undvika egenskaper som kan bedömas subjektivt, utan formulera istället egenskaper som kan bedömas utifrån konkreta exempel från t. ex. tidigare erfarenheter eller under aspningen.


\begin{document}

\begin{center}
\Huge{\bf{Krav- och meritprofil för Kärnstyret}}
\end{center}
\section*{Generella frågor för hela föreningen}
\subsection*{Beskriv kortfattat verksamheten}
Att på daglig basis driva sektionens verksamhet och dessutom ständigt jobba för att utveckla och förbättra sektionen. Styret är sektionens högst verkställande organ, med ansvar för att ekonomin sköts och att sektionens medlemmar får nytta av sektionens pengar. Kärnstyret hjälper till och finns som stöd för de andra föreningarna på sektionen.

\subsection*{Vilka typer av arbetsuppgifter ingår i er verksamhet?}
Man håller i och närvarar vid många möten. Deltar aktivt i frågor som berör sektionen och tar in andras åsikter. Dessutom håller man i sektionsmöten.

\subsection*{Vilka tidigare erfarenheter (arbete, utbildningar) ses som fördelaktiga?}
Tidigare meriterande erfarenheter är, i prioritetsordning:
\begin{enumerate}
\item Tidigare aktiv inom sektion/kåren
\item Tidigare styrelsearbete
\item Tidigare drivande inom idéellt arbete
\item Erfarenhet av att arbeta i grupp/projektledning
\end{enumerate}

\subsection*{Vilka personliga egenskaper ses som fördelaktiga?}
\begin{itemize}
\item Självständig - Kan hantera postspecifikt arbete fristående från de andra. Fatta egna beslut. Kunna hantera övriga aspekter av sitt liv.
\item Strukturerad - Kunna planera och hålla koll på sitt arbete, även när man har mycket att göra och är stressad.
\item Engagerad - Bry sig om och känna ett ansvar för sektionens verksamhet.
\item Ansvarstagande - Slutför uppgifter oavsett om det känns omotiverande.
\item Professionell - Kunna separera sina personliga åsikter och sitt privatliv från arbetet.
\item Se till alla sektionsmedlemmarnas bästa, inte bara de sektionsaktiva och de som hänger på Focus.
\item Strategisk - Ha ett långsiktigt och hållbart tänkande.
\item Förmåga att kommunicera med olika sorters människor
\item Kunna företräda någon annans åsikt.
\end{itemize}
\subsection*{Gruppegenskaper}
\begin{itemize}
\item Balans mellan visionärer och kortsiktig målmedvetenhet.
\item Balans mellan pragmatism och paragrafrytteri.
\item Personer som har varit aktiva i olika delar av verksamheten.
\item Mångsidig grupp.
\item Kunna visa förtroende för varandra.
\item Ha möjligthet att driva nya, mindre projekt. Antingen genom någon specifik person med sådant intresse eller framförallt som grupp i helhet.
\end{itemize}

\section*{Postspecifika egenskaper}

\subsection*{Ordförande}
\begin{itemize}
\item Kunna anpassa sin kommunikation - Mot studenter, lärare, etc.
\item Tidigare ledarerfarenheter.
\item Mötesvana - Kunna leda diskussioner och se till att folk inte pratar runt varandra.
\item Retorisk och duktig på att uttrycka sig i skrift.
\item Prestigelös - Att man inte söker ordförande för att vara just ordförande och för att bestämma. Att man inte vill vara ''chef'', utan ha en inspirerande roll.
\item Förtroendeingivande.
\item Strategisk - Kunna tänka långsiktigt och hållbart och se sektionens verksamhet i ett långsiktigt perspektiv.
\item Kunna hantera konflikter och obekväma sammanhang, och kunna ta kritik.
\item Kunna ta obekväma beslut och stå för dem, även om folk blir arga på
\end{itemize}

\subsection*{Vice ordförande}
\begin{itemize}
\item Tidigare sektionsengagemang - Ha god insikt i hur sektionen fungerar.
\item Kunna anpassa sin kommunikation - Mot studenter, lärare, etc.
\item Förtroendeingivande.
\item Tidigare ledarerfarenheter.
\item Våga fatta egna beslut.
\item Kunna hantera konflikter och obekväma sammanhang, och kunna ta kritik.
\item Kunna ta obekväma beslut och stå för dem, även om folk blir arga på en.

\end{itemize}

\subsection*{Kassör}
\begin{itemize}
\item Van vid självständigt arbete (t.ex. bokföring).
\item Ha disciplin.
\item Kunna se till medlemsnytta snarare än ekonomisk nytta.
\item Intresse för ekonomi.
\item Förtroendeingivande.
\item Ha erfarenhet av bokföring och ekonomiskt tänk.
\item Villig att hjälpa andra kassörer på sektionen med ekonomiarbete.
\item Noggrann.
\end{itemize}

\subsection*{Sekreterare}
\begin{itemize}
\item Tycker organisation och kontinuitet är viktigt.
\item Vara närvarande och uppmärksam.
\item Redo att införskaffa god kännedom om stadgar och reglemente.
\item Bra på att uttrycka sig i skrift. Tydlig, professionell och kärnfull.
\item Vara disciplinerad och ha god detaljkoll.
\end{itemize}

\subsection*{Skyddsombud}
\begin{itemize}
\item Bry sig om andra människor och deras studiesociala hälsa.
\item God förmåga att lyssna, visa empati och komma med vettiga kommentarer som passar situationen. Dock inte tro att man är en psykolog.
\item Behålla lugnet även i jobbiga och stressiga situationer. 
\item Inte ta åt sig av motgångar eller tråkiga händelser.
\item Inse när något berättas i förtroende, och inte sprida detta. 
\item Handlingskraftig
\item Något annat
\end{itemize}
        
\subsection*{Informationsansvarig}
\begin{itemize}
\item Ha intresse för kommunikations- och informationsarbete.
\item Duktig inom grafisk design.
\item Redo att bevaka sektionen och anpassa sin kommunikation därefter.
\item Duktig på att uttrycka sig i skrift, både informativt och intresseväckande.
\item Kunna ha övergripande ansvar för sektionens vecka, mastersmottagning och liknande projekt. 
\item Kunna planera projekt och delegera arbetsuppgifter.
\end{itemize}
                
\end{document}